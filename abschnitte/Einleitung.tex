\section{Einleitung}

\subsection{Problemstellung}

Die Scriptsprache Javascript, die im Jahre 1995 von Netscape entwickelt wurde, erfreute sich in den letzten Jahren enormer Popularität im Bereich der Webseiten- Entwicklung. Dort wird sie vor allem zur Verbesserung der Interaktivität von Webseiten genutzt. In jüngster Vergangenheit beginnt Javascript nun auch bei der serverseitigen Entwicklung eine Rolle zu spielen. In beiden Bereichen hat sich die Komplexität von Javascript-Programmen so weit gesteigert, dass sie mit jeder anderen Programmiersprache vergleichbar geworden ist. Damit werden natürlich verschiedene Software-Engineering-Methodiken, die sich im Laufe der Zeit über alle Sprachen hinweg etabliert haben, auch für Javascript interessant. Dabei müssen die Lösungen für Javascript allerdings einige Besonderheiten abdecken, die mit der Sprache einhergehen: So dient Javascript nach wie vor vor allem für die Erstellung interaktiver Webseiten, die in verschiedensten Browsern und Plattformen funktionieren sollen. In all diesen Browsern gibt es verschiedene Javascript-Umgebungen, die sich zum Teil syntaktisch unterscheiden. Auch bei der serverseitigen Verwendung von Javascript, also mit Javascript-Interpretern oder Headless-Browsern, gibt es verschiedene Javascript-Engines, die nicht in allen Punkten identisch sind und somit beim Testen berücksichtigt werden müssen.

\subsection{Zielsetzung}

Ziel dieser Seminararbeit ist es hierbei, auf den Bereich des Testens einzugehen und einen überblick über existierende Testframeworks zu geben.

\subsection{Vorgehensweise}

Zunächst werden dafür allgemeine Aspekte des Testens aufgezeigt. Anschließend werden diese genutzt, um verschiedene Tools und Ansätze für Javascript mit einander zu vergleichen. Anhand eines Fallbeispiels werden einige dieser Tools dann angewendet und veranschaulicht. Abschließend werden alle Ergebnisse zusammengefasst und ein Fazit gezogen.

\subsection{Zitate}

\cite{Google13} sagen hier steht ein Text. \\
\citet{Hidayat13} sagen hier steht ein Text. \\
\citet*{Yahoo13} sagen hier steht ein Text. \\
\citet*{Trostler13} sagen hier steht ein Text. \\
\citet*{Wiki13-01} sagen hier steht ein Text. \\
\citet*{Wiki13-02} sagen hier steht ein Text. \\
\citet*{Wiki13-03} sagen hier steht ein Text. \\
"`Hier steht ein Text."' \citep{Johansen10} \\
Hier steht ein Text. \citep[Vgl.][]{Kleivane11} \\
Hier steht ein Text. \citep[][S. 200]{Koch01} \\
Hier steht ein Text. \citep*[][S. 200]{Pivotal13} \\
Hier steht ein Text. \citep{Selenium13,Nguyen13} \\


