\section{Einleitung}

\subsection{Problemstellung}

Die Scriptsprache Javascript, die im Jahre 1995 von Netscape entwickelt wurde, erfreute sich in den letzten Jahren enormer Popularität im Bereich der Webseiten- Entwicklung \citep[Vgl.][]{Wiki13-02}. Dort wird sie vor allem zur Verbesserung der Interaktivität von Webseiten genutzt. Inzwischen werden aber auch ganze Anwendungen clientseitig im Browser mit HIlfe von Javascript betrieben. Ein prominentes Beispiel ist hier der E-Mail-Client Google Mail, dessen gesamte Anwendungsoberfläche auf Javascript aufsetzt \citep[Vgl.][]{Google13-02}. Seit einiger Zeit beginnt Javascript nun aber auch bei der serverseitigen Entwicklung eine Rolle zu spielen. Als großer Vertreter sei hier beispielhaft eBay genannt, die Teile ihrer Infrastrutur mit dem Javascript-Server node.js betreiben \citep[Vgl.][]{Ebay11}. In jüngster Vergangenheit entstehen auch viele hochinteraktive Browser-Spiele, z.B. auf Basis der Aves Engine, die sowohl clientseitig als auch serverseitig Javascript einsetzen \citep[Vgl.][]{Ihlenfeld13}. In beiden Bereichen hat sich die Komplexität von Javascript-Programmen damit so weit gesteigert, dass sie mit jeder anderen Programmiersprache vergleichbar geworden ist. Damit werden natürlich verschiedene Software-Engineering-Methodiken, die sich im Laufe der Zeit über alle Sprachen hinweg etabliert haben, auch für Javascript interessant. Dabei müssen die Lösungen für Javascript allerdings einige Besonderheiten abdecken, die mit der Sprache einhergehen: So dient Javascript nach wie vor vor allem für die Erstellung interaktiver Webseiten, die in verschiedensten Browsern und Plattformen funktionieren sollen. In all diesen Browsern gibt es verschiedene Javascript-Umgebungen, die sich zum Teil syntaktisch unterscheiden \citep[Vgl.][S. 17 ff]{Kleivane11}.

\subsection{Zielsetzung}

Das Testen von Software war schon immer ein wichtiger Bestandteil des Software-Engineerings \citep[Vgl.][S. 1]{Kleivane11}. Ziel dieser Seminararbeit ist es, auf diesen Teilbereich einzugehen und ihn vor allem in Hinblick auf die für die Sprache Javascript zur Verfügung stehenden Tools und Frameworks zu durchleuchten.

\subsection{Vorgehensweise}

Die Seminararbeit gliedert sich neben dieser Einleitung in vier weitere Abschnitte:
\begin{itemize}
	\item Zunächst wird im 2. Kapitel der Arbeit der aktuelle Stand der Formschung im Bereich der Softwaretests beschrieben, indem bewährte Testarten und Testmethodiken erläutert werden.
	\item Anschließend werden im 3. Kapitel eine Reihe konkreter Tools und Ansätze für die Javascript-Entwicklung aufgelistet und mit einander verglichen. Dabei wird vor allem darauf eingegangen, in wiefern diese die allgemeinen Konzepte des Testens umsetzen.
	\item Im 4. Kapitel werden dann Anhand eines Fallbeispiels einige der Tools angewendet.
	\item Im 5. und letzten Kapitel werden die Ergebnisse zusammengefasst und ein Fazit gezogen.
\end{itemize}

